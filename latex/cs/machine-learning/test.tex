\documentclass[a4paper,11pt]{book}
% Setting up the document with packages typical for mathematical texts
\usepackage{amsmath, amssymb, amsthm, mathtools}
\usepackage{geometry}
\geometry{margin=1in}
\usepackage{fancyhdr}
\usepackage{hyperref}
\usepackage{tocloft}
\usepackage{enumitem}

% Configuring fonts (using Latin Modern for compatibility)
\usepackage{lmodern}

% Defining custom theorem styles and environments
\theoremstyle{plain}
\newtheorem{theorem}{Theorem}[section]
\newtheorem{lemma}[theorem]{Lemma}
\newtheorem{proposition}[theorem]{Proposition}
\newtheorem{corollary}[theorem]{Corollary}
\theoremstyle{definition}
\newtheorem{definition}[theorem]{Definition}
\newtheorem{example}[theorem]{Example}

% Custom macros for algebraic notation
\newcommand{\Z}{\mathbb{Z}}
\newcommand{\Q}{\mathbb{Q}}
\newcommand{\R}{\mathbb{R}}
\newcommand{\C}{\mathbb{C}}
\newcommand{\F}{\mathbb{F}}
\newcommand{\Hom}{\mathrm{Hom}}
\newcommand{\Ker}{\mathrm{Ker}}
\newcommand{\Img}{\mathrm{Im}}
\newcommand{\id}{\mathrm{id}}

% Setting up header and footer
\pagestyle{fancy}
\fancyhf{}
\fancyhead[LE,RO]{\thepage}
\fancyhead[RE]{\nouppercase{\leftmark}}
\fancyhead[LO]{\nouppercase{\rightmark}}

% Configuring table of contents
\setlength{\cftchapindent}{0em}
\setlength{\cftsecindent}{2em}
\setlength{\cftsubsecindent}{4em}

\begin{document}

% Starting main content
\chapter{Groups}

\section{Introduction to Groups}

\begin{definition}
A \emph{group} is a set $G$ equipped with a binary operation $\cdot : G \times G \to G$ satisfying the following properties:
\begin{enumerate}[label=(\roman*)]
    \item \emph{Associativity}: For all $a, b, c \in G$, $(a \cdot b) \cdot c = a \cdot (b \cdot c)$.
    \item \emph{Identity}: There exists an element $e \in G$ such that for all $a \in G$, $e \cdot a = a \cdot e = a$.
    \item \emph{Inverses}: For each $a \in G$, there exists $b \in G$ such that $a \cdot b = b \cdot a = e$.
\end{enumerate}
If the operation is commutative (i.e., $a \cdot b = b \cdot a$ for all $a, b \in G$), the group is called \emph{abelian}.
\end{definition}

\begin{example}
The set of integers $\Z$ under addition forms an abelian group, with identity element $0$ and inverse $-a$ for each $a \in \Z$.
\end{example}

\begin{theorem}
Let $G$ be a group. The identity element $e \in G$ is unique.
\end{theorem}
\begin{proof}
Suppose $e$ and $e'$ are both identity elements. Then, for any $a \in G$, we have $a \cdot e = a$ and $e' \cdot a = a$. Consider $e \cdot e'$. Since $e$ is an identity, $e \cdot e' = e'$. Since $e'$ is an identity, $e \cdot e' = e$. Thus, $e = e'$.
\end{proof}

\section{Rings}

\begin{definition}
A \emph{ring} is a set $R$ equipped with two binary operations, addition ($+$) and multiplication ($\cdot$), satisfying:
\begin{enumerate}[label=(\roman*)]
    \item $(R, +)$ is an abelian group with identity $0$.
    \item Multiplication is associative: $(a \cdot b) \cdot c = a \cdot (b \cdot c)$ for all $a, b, c \in R$.
    \item Distributivity: $a \cdot (b + c) = a \cdot b + a \cdot c$ and $(b + c) \cdot a = b \cdot a + c \cdot a$ for all $a, b, c \in R$.
\end{enumerate}
A ring is \emph{commutative} if multiplication is commutative. A ring has a \emph{multiplicative identity} if there exists $1 \in R$ such that $1 \cdot a = a \cdot 1 = a$ for all $a \in R$.
\end{definition}

\begin{example}
The set of integers $\Z$ with standard addition and multiplication is a commutative ring with multiplicative identity $1$.
\end{example}

\section{Fields}

\begin{definition}
A \emph{field} is a commutative ring with a multiplicative identity $1 \neq 0$ in which every non-zero element $a \in F$ has a multiplicative inverse, i.e., there exists $b \in F$ such that $a \cdot b = 1$.
\end{definition}

\begin{example}
The sets $\Q$, $\R$, and $\C$ are fields under standard addition and multiplication. The set $\Z$ is not a field, as elements like $2$ have no multiplicative inverse in $\Z$.
\end{example}

\begin{proposition}
Every field is an integral domain, i.e., a commutative ring with $1 \neq 0$ and no zero divisors (if $a \cdot b = 0$, then $a = 0$ or $b = 0$).
\end{proposition}
\begin{proof}
Let $F$ be a field, and suppose $a \cdot b = 0$ with $a \neq 0$. Since $F$ is a field, $a$ has a multiplicative inverse $a^{-1}$. Multiply both sides of $a \cdot b = 0$ by $a^{-1}$: $a^{-1} \cdot (a \cdot b) = a^{-1} \cdot 0$. This gives $(a^{-1} \cdot a) \cdot b = 0$, so $1 \cdot b = 0$, hence $b = 0$. Thus, $F$ is an integral domain.
\end{proof}

\end{document}